\documentclass{ctexbook}
\usepackage[colorlinks=true]{hyperref}
\usepackage{geometry}
\usepackage{subcaption}
\usepackage{diagbox}
\setcounter{chapter}{-1}
\usepackage{mathtools}
\usepackage{amssymb,amsthm}
\newtheorem{proposition}{命题}[section]
\newtheorem{theorem}{定理}[section]
\theoremstyle{definition}
\newtheorem{definition}{定义}[section]
\newtheorem{example}{例}[section]
\theoremstyle{remark}
\newtheorem{remark}{注}
\everymath{\displaystyle}

\title{Elementary Analysis}
\author{Inverted Pastry}
\date{2023}

\begin{document}

\frontmatter
\maketitle
\tableofcontents

\mainmatter

\chapter{预备知识}

本章并非对于集合论等知识的严格叙述, 而是在读者对这些知识早已知悉的条件之下, 对后面所需的记号进行简单讨论.

\section{逻辑}

通用的逻辑符号大致有以下五种: $\neg$, $\wedge$, $\vee$, $\Rightarrow$, $\Leftrightarrow$, 它们分别表示非, 与, 或, 蕴含, 等价. 括号被用于划分复杂命题中的结构. 为简化书写, 如下由高至低约定逻辑符号的优先顺序: $\neg$, $\wedge$, $\vee$, $\Rightarrow$, $\Leftrightarrow$, 而他们的准确定义由下面的真值表给出. 其中 1 和 0 分别表示真假断言.

\begin{table}[htbp]
    \centering
    \begin{subtable}[htbp]{0.16\textwidth}
        \centering
        \begin{tabular}{|c|c|c|}
        \hline
        $A$ & 0 & 1 \\
        \hline
        $\neg{A}$ & 1 & 0 \\
        \hline
        \end{tabular}
        \caption{$\neg{A}$}
        \label{tab:1_a}
    \end{subtable}
    \begin{subtable}[htbp]{0.2\textwidth}
        \centering
        \begin{tabular}{|c|c|c|}
        \hline
        \diagbox{A}{B} & 0 & 1 \\
        \hline
        0 & 0 & 0 \\
        \hline
        1 & 0 & 1 \\
        \hline
        \end{tabular}
        \caption{$A\wedge{B}$}
        \label{tab:1_b}
    \end{subtable}
    \begin{subtable}[htbp]{0.2\textwidth}
        \centering
        \begin{tabular}{|c|c|c|}
        \hline
        \diagbox{A}{B} & 0 & 1 \\
        \hline
        0 & 0 & 1 \\
        \hline
        1 & 1 & 1 \\
        \hline
        \end{tabular}
        \caption{$A\vee{B}$}
        \label{tab:1_c}
    \end{subtable}
    \begin{subtable}[htbp]{0.2\textwidth}
        \centering
        \begin{tabular}{|c|c|c|}
        \hline
        \diagbox{A}{B} & 0 & 1 \\
        \hline
        0 & 1 & 1 \\
        \hline
        1 & 0 & 1 \\
        \hline
        \end{tabular}
        \caption{$A\Rightarrow{B}$}
        \label{tab:1_d}
    \end{subtable}
    \begin{subtable}[htbp]{0.2\textwidth}
        \centering
        \begin{tabular}{|c|c|c|}
        \hline
        \diagbox{A}{B} & 0 & 1 \\
        \hline
        0 & 1 & 0 \\
        \hline
        1 & 0 & 1 \\
        \hline
        \end{tabular}
        \caption{$A\Leftrightarrow{B}$}
        \label{tab:1_e}
    \end{subtable}
    \caption{真值表}
    \label{tab:1}
\end{table}

值得注意的是, 假命题蕴含任何命题, $A$ 或 $B$ 为真不要求 $A$, $B$ 中至多一者为真. 而与后者相对应的另一不常用逻辑符号 $\veebar$ (异或) 则有如下真值表.

\begin{table}[htbp]
    \centering
    \begin{tabular}{|c|c|c|}
        \hline
        \diagbox{A}{B} & 0 & 1 \\
        \hline
        0 & 0 & 1 \\
        \hline
        1 & 1 & 0 \\
        \hline
        \end{tabular}
        \caption{$A\veebar{B}$}
    \label{tab:2}
\end{table}

运用上述定义则可由穷举法证得如下命题.

\begin{proposition}
    \begin{enumerate}
        \item $\neg(A\wedge{B})\Leftrightarrow{\neg{A}\vee\neg{B}}$;
        \item $\neg(A\vee{B})\Leftrightarrow{\neg{A}\wedge\neg{B}}$;
        \item $(A\Rightarrow{B})\Leftrightarrow(\neg{B}\Rightarrow\neg{A})$;
        \item $(A\Rightarrow{B})\Leftrightarrow(\neg{A}\vee{B})$;
        \item $\neg(A\Rightarrow{B})\Leftrightarrow(A\wedge\neg{B})$.
    \end{enumerate}
\end{proposition}

由此可见, 仅用非和与即可定义其他提及的逻辑符号.

\begin{enumerate}
    \item $A\vee{B}$ 定义为 $\neg(\neg{A}\wedge\neg{B})$;
    \item $A\Rightarrow{B}$ 定义为 $\neg{A}\vee{B}$;
    \item $A\Leftrightarrow{B}$ 定义为 $(A\Rightarrow{B})\wedge(B\Rightarrow{A})$;
    \item $A\veebar{B}$ 定义为 $\neg(A\Leftrightarrow{B})$.
\end{enumerate}


至此, 我们已经对认知中的逻辑建立起了较为严格的叙述.

\section{证明}

命题的证明方法不计其数, 但总的来说可以归结为对 $A\Rightarrow{C}$ 的处理, 其中 $A$, $C$ 分别是命题的假设和结论. 下面列出几种较为常见的证明方法.

\subsection{直接证明}

直接证明的过程是对一系列既定事实的组合, 而不做任何进一步的假设, 也即证明命题 $A\Rightarrow{C}$ 的等价命题 $(A\Rightarrow{B_1})\wedge(B_1\Rightarrow{B_2})\wedge\cdots\wedge(B_{n-1}\Rightarrow{B_n})\wedge(B_n\Rightarrow{C})\wedge$, 后者简记为 $A\Rightarrow{B_1}\Rightarrow{B_2}\Rightarrow\cdots\Rightarrow{B_n}\Rightarrow{C}$. 该方法的原理为: 若 $A$ 与 $A\Rightarrow{B}$ 均为真, 则 $B$ 为真. 下举一例.

\begin{example}
    若 $a$ 为奇数, 则 $a^2$ 也为奇数.

    \begin{proof}
        $a$ 为奇数意味着 $\exists{k}\left((k\in\mathbb{Z})\wedge(a=2k+1)\right)$, 于是 $a^2=(2k+1)^2=4k^2+4k+1=2\left(2k^2+2k\right)+1$ 亦为奇数.
    \end{proof}
\end{example}

\subsection{换质位法}

同时否定一个命题的前件和后件, 其结果称为原命题的换质; 交换一个命题前件和后件的位置, 其结果称为原命题的换位. 换质位法即利用换质与换位将一个命题改为了与其等价的逆否命题.

\begin{example}
    考虑集合 $S$ 的划分 $!!$, 若 $|S|>100$ 且 $|A|=10$, 则存在 $\alpha_0\in{A}$ 使得 $\left|S_{\alpha_0}\right|>10$.
\end{example}

\end{document}
