\documentclass{ctexbook}
\usepackage[colorlinks=true]{hyperref}
\usepackage{geometry}
\usepackage{subcaption}
\usepackage{diagbox}
\usepackage{mathtools}
\usepackage{amssymb,amsthm}
\newtheorem{theorem}{定理}[section]
\newtheorem{lemma}{引理}[section]
\newtheorem{corollary}{推论}[section]
\newtheorem{proposition}{命题}[section]
\theoremstyle{definition}
\newtheorem{definition}{定义}[section]
\newtheorem{example}{例}[section]
\theoremstyle{remark}
\newtheorem{remark}{注}
\everymath{\displaystyle}
\usepackage{float}

\title{数学分析}
\author{倒置的糕点}
\date{2023}

\begin{document}

\frontmatter
\maketitle
\tableofcontents

\mainmatter

\chapter{逻辑, 集合与关系}

本章并非对于逻辑, 集合与关系等知识的严格叙述, 而是在读者对此早已知悉的条件之下, 对后面所需的记号进行简单讨论. 读者应跳读本章, 有需要时再回来查阅具体内容.

\section{逻辑}

通用的逻辑符号大致有五种: $\neg$, $\wedge$, $\vee$, $\Rightarrow$ 和 $\Leftrightarrow$, 它们分别表示非, 与, 或, 蕴含以及等价, 准确定义由下面的真值表给出. 其中 1 和 0 分别表示逻辑学中的真, 假命题\footnote{现阶段, 我们认为命题可以明确地判断真假, 且任何可以明确判断真假者均为命题. 例如, ``本句不为命题" 是命题且是假命题, 而 ``本句为假命题" 则不是命题.}\footnote{数学中的命题是一个不太重要的定理, 又或者是一个非常基本或显而易见的定理, 以至于无需证明即可陈述. 这里应予以区分.}.

\begin{table}[H]
    \centering
    \begin{subtable}[H]{0.16\textwidth}
        \centering
        \begin{tabular}{|c|c|c|}
        \hline
        $A$ & 0 & 1 \\
        \hline
        $\neg{A}$ & 1 & 0 \\
        \hline
        \end{tabular}
        \caption{$\neg{A}$}
        \label{tab:1.1_a}
    \end{subtable}
    \begin{subtable}[H]{0.2\textwidth}
        \centering
        \begin{tabular}{|c|c|c|}
        \hline
        \diagbox{A}{B} & 0 & 1 \\
        \hline
        0 & 0 & 0 \\
        \hline
        1 & 0 & 1 \\
        \hline
        \end{tabular}
        \caption{$A\wedge{B}$}
        \label{tab:1.1_b}
    \end{subtable}
    \begin{subtable}[H]{0.2\textwidth}
        \centering
        \begin{tabular}{|c|c|c|}
        \hline
        \diagbox{A}{B} & 0 & 1 \\
        \hline
        0 & 0 & 1 \\
        \hline
        1 & 1 & 1 \\
        \hline
        \end{tabular}
        \caption{$A\vee{B}$}
        \label{tab:1.1_c}
    \end{subtable}
    \begin{subtable}[H]{0.2\textwidth}
        \centering
        \begin{tabular}{|c|c|c|}
        \hline
        \diagbox{A}{B} & 0 & 1 \\
        \hline
        0 & 1 & 1 \\
        \hline
        1 & 0 & 1 \\
        \hline
        \end{tabular}
        \caption{$A\Rightarrow{B}$}
        \label{tab:1.1_d}
    \end{subtable}
    \begin{subtable}[H]{0.2\textwidth}
        \centering
        \begin{tabular}{|c|c|c|}
        \hline
        \diagbox{A}{B} & 0 & 1 \\
        \hline
        0 & 1 & 0 \\
        \hline
        1 & 0 & 1 \\
        \hline
        \end{tabular}
        \caption{$A\Leftrightarrow{B}$}
        \label{tab:1.1_e}
    \end{subtable}
    \caption{真值表}
    \label{tab:1.1}
\end{table}

值得注意的是, 假命题蕴含任何命题; $A$ 或 $B$ 为真不要求 $A$, $B$ 中恰有一者为真. 相对应地, 有另一不常用逻辑符号 $\veebar$ (异或), 其真值表如下.

\begin{table}[H]
    \centering
    \begin{tabular}{|c|c|c|}
        \hline
        \diagbox{A}{B} & 0 & 1 \\
        \hline
        0 & 0 & 1 \\
        \hline
        1 & 1 & 0 \\
        \hline
    \end{tabular}
    \caption{$A\veebar{B}$}
    \label{tab:1.2}
\end{table}

对于更加复杂的命题, 括号被用于划分结构. 但为简化书写, 由高至低约定逻辑符号的优先顺序: $\neg$, $\wedge$, $\vee$, $\Rightarrow$, $\Leftrightarrow$.

运用上述定义则可由穷举法证得如下命题.

\begin{proposition}
    \begin{enumerate}
        \item $\neg(A\wedge{B})\Leftrightarrow{\neg{A}\vee\neg{B}}$;
        \item $\neg(A\vee{B})\Leftrightarrow{\neg{A}\wedge\neg{B}}$;
        \item $(A\Rightarrow{B})\Leftrightarrow(\neg{B}\Rightarrow\neg{A})$;
        \item $(A\Rightarrow{B})\Leftrightarrow(\neg{A}\vee{B})$;
        \item $\neg(A\Rightarrow{B})\Leftrightarrow(A\wedge\neg{B})$.
    \end{enumerate}
\end{proposition}

由此可见, 仅用非和与即可定义其他提及的逻辑符号.

\begin{enumerate}
    \item $A\vee{B}$ 定义为 $\neg(\neg{A}\wedge\neg{B})$;
    \item $A\Rightarrow{B}$ 定义为 $\neg{A}\vee{B}$;
    \item $A\Leftrightarrow{B}$ 定义为 $(A\Rightarrow{B})\wedge(B\Rightarrow{A})$;
    \item $A\veebar{B}$ 定义为 $\neg(A\Leftrightarrow{B})$.
\end{enumerate}

至此, 我们已经对认知中的逻辑建立起了较为严格的叙述.

\section{集合}

\section{关系}

\subsection{等价关系}

\subsection{映射}

\subsection{序关系}

\chapter{实数}

\end{document}
